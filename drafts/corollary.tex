%% LyX 2.4.2.1 created this file.  For more info, see https://www.lyx.org/.
%% Do not edit unless you really know what you are doing.
\documentclass[american]{article}
\usepackage[T1]{fontenc}
\usepackage[utf8]{inputenc}
\usepackage{amsmath}
\usepackage{babel}
\begin{document}
		The singularities in the reduced dynamics play a crucial role in the
		loss of controlled invariance of the set $\Gamma$. For a trajectory
		$q_{*}\left(t\right)$ of a general mechanical system~(1) the vanishing
		of the coefficient $\alpha\left(\theta_{*}\left(t_{s}\right)\right)$
		in corresponding reduced dynamics along a parametrization $q_{*}\left(t\right)=\phi\left(\theta_{*}\left(t\right)\right)$
		imposes a specific constraint on the derivative $\dot{\theta}_{s}$.
		In particular, under the assumption that namely $\theta_{*}\left(t\right)$
		is smooth, the following relation must hold:
		\begin{equation}
		\beta\left(\theta_{s}\right)\dot{\theta}_{s}^{2}+\gamma\left(\theta_{s}\right)=0,\label{eq:constr-for-dtheta}
		\end{equation}
		If this equation is solvable -- that is, if coefficients $\beta\left(\theta_{s}\right)$
		and $\gamma\left(\theta_{s}\right)$ have opposite signs -- then
		the mechanical system at the configuration $q_{s}=\phi\left(\theta_{s}\right)$
		can move along the direction $\phi'\left(\theta_{s}\right)$, but
		only with a specific velocity given by:
		\[
		\dot{q}_{s}=\pm\sqrt{-\frac{\gamma\left(\theta_{s}\right)}{\beta\left(\theta_{s}\right)}}\phi'\left(\theta_{s}\right).
		\]
		A similar result is obtained when considering an implicit VHC $h\left(q_{*}\left(t\right)\right)\equiv0$.
		In this case, the admissible velocity $\dot{q}_{s}$ can be determined
		by imposing the conditions
		\[
		\frac{d}{dt}h\left(q_{*}\left(t\right)\right)=0,\quad\frac{d^{2}}{dt^{2}}h\left(q_{*}\left(t\right)\right)=0\quad\text{at}\,t=t_{s}.
		\]
		These conditions lead to a system of two equations with respect to
		$\dot{q}_{s}$ 
		\begin{align}
		NL_{\dot{q}_{s}}^{2}h\left(q_{s}\right)-Ndh\left(q_{s}\right)C\left(q_{s},\dot{q}_{s}\right)\dot{q}_{s} & =Ndh\left(q_{s}\right)G\left(q_{s}\right)\label{eq:condition-for-dq}\\
		Ndh\left(q_{s}\right)\dot{q}_{s} & =0,\nonumber 
		\end{align}
		where $N\in\ker\left[dh\left(q_{s}\right)B\left(q_{s}\right)\right]^{\top}\setminus\left\{ 0\right\} $.
		Substituting $\dot{q}_{s}=\dot{\theta}_{s}V$, where $V\in\ker\left[Ndh\left(q_{s}\right)\right]\setminus\left\{ 0\right\} $
		is a constant, treduces the system to a quadratic equation in $\dot{\theta}_{s}$:
		\[
		N\left(L_{V}^{2}h\left(q_{s}\right)-dh\left(q_{s}\right)C\left(q_{s},V\right)V\right)\dot{\theta}_{s}^{2}=Ndh\left(q_{s}\right)G\left(q_{s}\right)
		\]
		which is equivalent to equation~(\ref{eq:constr-for-dtheta}). The
		condition $B_{\perp}\left(q_{s}\right)G\left(q_{s}\right)\ne0$ in
		Theorem\textasciitilde\textbackslash ref\{thm:singularity-necessity\}
		implies that $Ndh\left(q_{s}\right)G\left(q_{s}\right)\ne0$. Consequently,
		equation\textasciitilde (\textbackslash ref\{eq:condition-for-dq\})
		either admits two solutions of opposite signs or has no solution at
		all. The former case permits the existence of smooth trajectories
		$q_{*}\left(t\right)$, as demonstrated in section above, but destroys
		the structure of the controlled invariant manifold. Therefore, Definition\textasciitilde\textbackslash ref\{def:VHC\}
		excludes from consideration all trajectories described in Theorem\textasciitilde\textbackslash ref\{thm:singularity-necessity\}.
\end{document}
