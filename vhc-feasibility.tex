\documentclass[letterpaper, 10 pt, conference]{ieeeconf}
\IEEEoverridecommandlockouts
\overrideIEEEmargins

% \documentclass[journal,twoside,web]{ieeecolor}
% \usepackage{lcsys}
% \usepackage{cite}

\usepackage{amsmath}
\usepackage{amssymb}
\usepackage{mathptmx}
\usepackage{graphics}
\usepackage{graphicx}
\usepackage[utf8]{inputenc}
\usepackage[T1]{fontenc}

\let\proof\relax
\let\endproof\relax
\usepackage{amsthm}
\usepackage{url}


\theoremstyle{plain}
\newtheorem{thm}{\protect\theoremname}

\newcounter{definition}
\theoremstyle{definition}
\newtheorem{defn}[definition]{\protect\definitionname}

\newcounter{remark}
\theoremstyle{remark}
\newtheorem{rem}[remark]{\protect\remarkname}

\providecommand{\definitionname}{Definition}
\providecommand{\lemmaname}{Lemma}
\providecommand{\remarkname}{Remark}
\providecommand{\theoremname}{Theorem}

\title{
    Virtual Holonomic Constraints in Motion Planning: \\
    Revisiting Feasibility and Limitations
}

\author{
  Maksim Surov$^{1}$
  % \thanks{
  %   *This work was not supported by any organization
  % }
  \thanks{
    $^{1}$Maksim Surov is with Department of Information Technologies and AI, 
    Sirius University of Science and Technology, 
    Sochi, Russia
    {\tt\small surov.m.o@gmail.com}
  }
}

\begin{document}
  \maketitle
  \thispagestyle{empty}
  \pagestyle{empty}

  \begin{abstract}
    This paper addresses the feasibility of virtual holonomic constraints (VHCs) 
    in the context of motion planning for underactuated mechanical systems with a single degree of underactuation.
    While existing literature has established a widely accepted definition of VHC,
    we argue that this definition is overly restrictive and excludes a broad class of admissible trajectories from consideration.
    To illustrate this point, we analyze a periodic motion of the Planar Vertical Take-Off and Landing (PVTOL) aircraft.
    The corresponding phase trajectory and reference control input are analytic functions.
    We demonstrate the stabilizability of this solution by constructing a 
    feedback controller that ensures asymptotic orbital stability.
    However, for this solution -- as well as for a broad class of similar ones -- 
    there exists no VHC that satisfies the conventional definition.
    This observation calls for a reconsideration of how the notion of VHC is defined,
    with the potential to significantly expand the practical applicability of VHCs in motion planning.
  \end{abstract}

  \section{Introduction}
    
    In this paper, we consider Euler-Lagrange systems described by the
    equation
    \begin{equation}
      M\left(q\right)\ddot{q}+C\left(q,\dot{q}\right)\dot{q}+G\left(q\right)=B\left(q\right)u\label{eq:lag-sys}
    \end{equation}
    where $q\in\mathbb{R}^{n}$ denotes the generalized coordinates and
    $u\in\mathbb{R}^{n-1}$ represents the control inputs. For simplicity,
    we assume the configuration space is $\mathbb{R}^{n}$, thereby avoiding
    a discussion of its topological properties. The matrix $M\left(q\right)\in\mathbb{R}^{n\times n}$
    is positive definite; $C\left(q,\dot{q}\right)\in\mathbb{R}^{n\times n}$
    is linear in $\dot{q}$; $G\left(q\right)\in\mathbb{R}^{n}$ is a
    vector; $B\left(q\right)\in\mathbb{R}^{n\times\left(n-1\right)}$
    is of rank $n-1$. All functions are assumed to be continuously differentiable.
    This system is underactuated with underactuation degree one.
    
    For system~(\ref{eq:lag-sys}), we consider the problem of motion
    planning, which involves finding smooth functions $q_{*}\left(t\right)\in C^{2}\left(\mathbb{R}\right)$
    and $u_{*}\left(t\right)\in C^{0}\left(\mathbb{R}\right)$ that satisfy
    the system dynamics. While specific applications may impose additional
    requirements on these functions, a typical criterion is the existence of stabilizing feedback
    which ensures that the planned motion can be executed on a physical system.

    The problem of motion planning has been studied in a number of publications~\cite{Shiriaev-2005-constructive-tool,Surov-2015,Freidovich-2007,Maggiore-2013,Consolini-2011}
    using the \textbf{virtual holonomic constraints} (VHCs) approach.
    This method assumes that the desired trajectory $q_{*}\left(t\right)$
    satisfies a geometric constraint of the form $h\left(q_{*}\left(t\right)\right)=0$,
    where $h:\mathbb{R}^{n}\to\mathbb{R}^{n-1}$ is a smooth function
    with Jacobian $dh\left(q\right)\equiv\frac{\partial h\left(q\right)}{\partial q}$
    of rank $n-1$ (see, e.g., \cite{Maggiore-2013,Otsason-2019}). Alternatively,
    the VHC can be defined in parametric form $q_{*}\left(t\right)=\phi\left(\theta_{*}\left(t\right)\right)$,
    where $\theta$ is a scalar parameter and $\phi:\mathbb{R}\to\mathbb{R}^{n}$
    is a smooth function. As shown in~\cite{Shiriaev-2005-constructive-tool},
    the VHC framework in application to motion planning enables reduction of the original $2n$-dimensional
    dynamics~(\ref{eq:lag-sys}) to a single scalar second-order differential
    equation, known as the \textbf{reduced dynamics}. A solution $\theta_{*}\left(t\right)$
    to this equation automatically defines corresponding solution
    $q_{*}\left(t\right)=\phi\left(\theta_{*}\left(t\right)\right)$ of the
    system~(\ref{eq:lag-sys}). This technique has significantly
    simplified the motion planning problem and has facilitated the solution
    of many challenging control tasks~\cite{Surov-2015,Freidovich-2007,Buss-2016}. 
    
    In~\cite{Maggiore-2013}, the authors initiated a detailed discussion
    on the definition of VHC and the corresponding necessary conditions
    on the functions $\phi\left(\cdot\right)$ and $h\left(\cdot\right)$.
    To this end, they relate the VHCs concept to the notion of \textbf{controlled
    invariant manifolds}~\cite[Chapter 6.1]{Isidori-1995}. For clarity
    and brevity, we present a slightly modified version of the definition, adapted from \cite[Definition 2.1]{Maggiore-2013}:
    \begin{defn}
    \label{def:VHC}
      A virtual holonomic constraint of order $n-1$ is
      a relation $h\left(q\right)=0$, where $h:\mathbb{R}^{n}\to\mathbb{R}^{n-1}$
      is smooth, $\mathrm{rank}\,dh\left(q\right)=n-1$ for all
      $q\in h^{-1}\left(0\right)$ and the set 
      \[
          \Gamma=\left\{ \left(q,\dot{q}\right):h\left(q\right)=0\quad\text{and}\quad\frac{\partial h\left(q\right)}{\partial q}\dot{q}=0\right\} 
      \]
      is \textbf{controlled invariant}. That is, there exists a smooth feedback
      $u\left(q,\dot{q}\right)$ such that $\Gamma$ is positively invariant
      for the closed-loop system. The set $\Gamma$ is called the \textbf{constraint
      manifold} associated with the VHC $h\left(q\right)=0$.
    \end{defn}
    %
    Similar definitions can be found in other publications including~\cite{Consolini-2011,Otsason-2019,Consolini-2010,Jankuloski-2012,Consolini-2018}.
    In some of these works~\cite{Consolini-2011,Consolini-2010,Jankuloski-2012}, the term \textit{feasible} VHC is used. A common requirement
    across all these publications is that the entire set $\Gamma$ be
    controlled invariant. Moreover, the authors underline, that the loss
    of controlled invariance implies that the VHC is \textit{not feasible}
    \cite[Chapter 2]{Jankuloski-2012}.

    In this work, we continue the discussion on the definition of VHCs.
    We argue that requiring the set $\Gamma$ to be \textbf{controlled
    invariant is unnecessarily restrictive} in the context of motion planning,
    as it excludes a broad class of viable solutions. 
    %
    While the existence of the structure of the controlled invariant manifold 
    may be beneficial in certain applications -- such
    as the design of stabilizing feedback controllers -- 
    we demonstrate that this condition is by no means necessary.
    In particular, the absence of controlled invariance does not preclude 
    the possibility of trajectory stabilization. 
    Furthermore, limiting our analysis to VHCs that satisfy Definition~\ref{def:VHC} 
    may result in a fundamental misjudgment of the broader set of stabilizable trajectories.
    % может сузить класс

    In Section~\ref{sec:pvtol-trajectory}, we examine an example of a periodic trajectory of the PVTOL aircraft and prove that there {\bf exists no VHC} satisfying
    Definition~\ref{def:VHC} associated with this trajectory. For this trajectory we also demonstrate the existence of a state feedback providing orbital asymptotic stability.
    In Section~\ref{sec:non-regular-vhc}, we formulate Theorem~\ref{thm:singularity-necessity} that establishes the necessity of the failure of the controlled invariance property for a certain class of solutions. We also present a class of trajectories of the PVTOL system for which Definition~\ref{def:VHC} fails. The concluding remarks and suggestions for a more precise definition are given in Section~\ref{sec:conclusion}.

  \subsection*{Notation}
    In accordance with \cite{Isidori-1995}, we introduce the following
    conventions:
    \begin{itemize}
      \item 
        The directional derivative of a smooth function $h:\mathbb{R}^{n}\to\mathbb{R}^{k}$
        along a vector field $f\in \mathbb{R}^{n}$ is denoted by $L_{f}h\left(q\right)\equiv\frac{\partial h\left(q\right)}{\partial q}f\left(q\right)\in\mathbb{R}^{k}$.
      \item
        The higher-order directional derivatives are defined recursively as:
        $L_{f}^{l}h\left(q\right)\equiv L_{f}L_{f}^{l-1}h\left(q\right)\in\mathbb{R}^{k}$.
      \item
        The Jacobian matrix of a function $h:\mathbb{R}^{n}\to\mathbb{R}^{k}$
        is denoted by $dh\left(q\right)\equiv\frac{\partial h\left(q\right)}{\partial q}\in\mathbb{R}^{k\times n}$.
      \item 
        The column space of a matrix $A$ is denoted by $\mathrm{Im}\left[A\right]$. 
      \item
        The kernel of a matrix $A\in\mathbb{R}^{n\times m}$ is defined as: $\ker\left[ A\right]\equiv\left\{ x\in\mathbb{R}^{m}\mid Ax=0\right\} .$
    \end{itemize}

  \section{Motivating Example: Periodic Trajectory of a PVTOL Aircraft}
  \label{sec:pvtol-trajectory}

    \begin{figure}
      \begin{centering}
        \vspace{4pt}
        \includegraphics[width=8.5cm]{fig/pvtol_gen_coord_and_tictoc}
      \par\end{centering}
      \caption{PVTOL aircraft: generalized coordinates and tic-toc maneuver schematically.}
      \label{fig:pvtol-tic-toc}
    \end{figure}

    Let us consider the PVTOL aircraft~\cite{Hauser-1992-pvtol},
    the dynamics of which are governed by the system of ordinary differential equations:
    \begin{equation}
    \label{eq:pvtol-dynamics}
      \ddot{x}=-\sin{\psi} \, u_{1},
      \quad\ddot{z}=\cos{\psi} \, u_{1} - 1,
      \quad\ddot{\psi} = u_{2},
    \end{equation}
    where $q=\left(x,z,\psi\right)^{\top}\in\mathbb{R}^{3}$ are generalized
    coordinates and $u=\left(u_{1},u_{2}\right)^{\top}\in\mathbb{R}^{2}$
    represents the control inputs, which are allowed to take both positive and negative values.
    The coordinates $x,z$ specify the position of aircraft's center of mass, while the angle $\psi$ defines its attitude, as illustrated in Fig.~\ref{fig:pvtol-tic-toc}.
    It is straightforward to verify that the PVTOL dynamics are of the Euler-Lagrange type, with the following matrices: 
    $M = I_{3\times3}$, $C = 0_{3\times3}$, 
    \[
      G = \left(\begin{array}{c}
          0\\
          1\\
          0
      \end{array}\right)
      \quad\text{and}\quad
      B\left(q\right)=\left(\begin{array}{cc}
          -\sin\psi & 0\\
          \cos\psi & 0\\
          0 & 1
      \end{array}\right).
    \]
    In this notation, the system can be rewritten compactly as
    $$
        \ddot{q}=B\left(q\right)u-G,
    $$
    which aligns with the general structure of Euler--Lagrange systems~(\ref{eq:lag-sys}).
    
  \subsection{Tic-Toc Maneuver of PVTOL Aircraft}
    Let us consider two analytic functions
    \begin{align}
    \label{eq:ref-traj}
      q_{*}\left(t\right) & =\left(\begin{array}{c}
          \sin t\\
          -\frac{1}{2}\sin^{2}t\\
          \frac{\pi}{2}-\arctan\left(2\sin t\right)
          \end{array}\right) \in C^\omega(\mathbb{R}) \quad\text{and}\\
      u_{*}\left(t\right) & =\left(\begin{array}{c}
          \sin t\sqrt{4\sin^{2}t+1}\\
          \frac{12\sin t+2\sin3t}{\left(3-2\cos2t\right)^{2}}
          \end{array}\right)  \in C^\omega(\mathbb{R}).\nonumber 
    \end{align}
    It can be readily verified by direct substitution that the pair $\left(q_{*}\left(t\right),u_{*}\left(t\right)\right)$
    satisfies equations~(\ref{eq:pvtol-dynamics}). 
    The trajectory
    $q_{*}\left(t\right)$ corresponds to the so-called \textit{tic-toc} aerobatic maneuver of the aircraft, which is schematically illustrated in Fig.~\ref{fig:pvtol-tic-toc}.
    In the following, we show that \textbf{no VHC} 
    satisfying Definition~\ref{def:VHC}
    can be associated with this trajectory.

  \subsection{The Absence of Controlled Invariance}
    Suppose, for the sake of contradiction, that there exists a function 
    $h:C^{2}\left(\mathbb{R}^{3}\to\mathbb{R}^{2}\right)$
    satisfying the following conditions:
    \begin{itemize}
      \item 
        $h\left(q_{*}\left(t\right)\right)\equiv0$ for all $t$,
      \item 
        $\mathrm{rank}\,dh\left(q\right)=2$ for all $q \in h^{-1}\left(0\right)$,
      \item 
        the set
        \[
          \Gamma\equiv\left\{ \left(q,\dot{q}\right)\in\mathbb{R}^{6}\mid h\left(q\right)=0
           \,\,\text{and}\,\, L_{\dot{q}}h\left(q\right)=0\right\} 
        \]
        is a controlled invariant manifold.
    \end{itemize}
    As discussed in~\cite{Consolini-2018}, the set $\Gamma$ is a two-dimensional manifold, which represents the tangent bundle $\Gamma=T\mathcal{C}$,
    where the base space $\mathcal{C}$ is defined as 
    $$
      \mathcal{C}=\left\{ q\in\mathbb{R}^{3}\mid\exists\,t:\, q=q_{*}\left(t\right)\right\}.
    $$

    To establish the controlled invariance of $\Gamma$, we analyze the dynamics of the variable $y=h\left(q\right)$, given
    by 
    \begin{align}
    \label{eq:ddot_y}
      \dot{y} & =L_{\dot{q}}h\left(q\right)\nonumber \\
      \ddot{y} & =dh\left(q\right)\left(B\left(q\right)u-G\right)+L_{\dot{q}}^{2}h\left(q\right).
    \end{align}
    According to~\cite[Chapter 6]{Isidori-1995}, the set $\Gamma$ is controlled invariant if and only if, for every $\left(q,\dot{q}\right)\in\Gamma$,
    there exists a smooth mapping $u:\Gamma\to\mathbb{R}^{2}$ such that
    the right hand side of the equation~(\ref{eq:ddot_y}) is identically
    zero. This requirement implies that the following system of linear algebraic equations in~$u$ must be solvable for all $\left(q,\dot{q}\right)\in\Gamma$:
    \begin{equation}
    \label{eq:linsys-for-u}
      dh\left(q\right)B\left(q\right)u=dh\left(q\right)G-L_{\dot{q}}^{2}h\left(q\right).
    \end{equation}
    Here, two options are possible:
    \begin{itemize}
      \item 
        The matrix $dh\left(q\right)B\left(q\right) \in \mathbb{R}^{(n-1)\times (n-1)}$ is of full rank. In
        this case, $\Gamma$ is controlled invariant for any bounded right-hand side.
      \item 
        If the matrix $dh\left(q\right)B\left(q\right)$ is not of full rank
        at some $q_{0}\in\mathcal{C}$, then there exists a nonzero row
        vector $N$ such that $N dh\left(q\right)B\left(q\right)=0$. In this case, the necessary condition for controlled invariance becomes
        \begin{equation}
        \label{eq:dq_constr}
            Ndh\left(q_{0}\right)G-NL_{\dot{q}}^{2}h\left(q_{0}\right)=0\quad\text{for all}\quad\dot{q}\in T_{q_{0}}\mathcal{C}.
        \end{equation}
        Importantly, the first term $Ndh\left(q_{0}\right)G$ does not depend
        on $\dot{q}$, while the second term is a quadratic form in $\dot{q}$. This implies that the only way to satisfy condition~(\ref{eq:dq_constr})  is for both terms to vanish: $Ndh\left(q_{0}\right)G=0$ and $NL_{\dot{q}}^{2}h\left(q_{0}\right)=0$
        for all $\dot{q}\in T_{q_{0}}\mathcal{C}$.
    \end{itemize}
    Let us consider the point $q_{0}\in\mathcal{C}$ given by:
    $$
      q_{0}=q_{*}\left(0\right)=\left(0,0,\frac{\pi}{2}\right)^{\top}.
    $$
    At this point, the velocity vector $\dot{q}_{0}=\dot{q}_{*}\left(0\right)=\left(1,0,-2\right)^{\top}\in T_{q_{0}}\mathcal{C}$
    lies in the image of the matrix
    \begin{align*}
      B\left(q_{0}\right) & =\left(\begin{array}{cc}
      -1 & 0\\
      0 & 0\\
      0 & 1
      \end{array}\right), \quad \text{that is,} \quad
      \dot{q}_{0}\in\mathrm{Im}\left[B\left(q_{0}\right)\right] .
    \end{align*}
    The condition $h\left(q_{*}\left(t\right)\right)\equiv0$ implies
    that the rows of $dh\left(q_{0}\right)$ are orthogonal to $\dot{q}_{0}$:
    $dh\left(q_{0}\right)\dot{q}_{0}=0$. 
    Consequently, the matrix $dh\left(q_{0}\right)B\left(q_{0}\right)$ cannot have full rank, hence 
    it possesses a left annihilator $N\in\ker\left[dh\left(q_{0}\right)B\left(q_{0}\right)\right]^{\top}\setminus\left\{ 0\right\}$.
    Then, in
    order for $\Gamma$ to be controlled invariant, the condition~(\ref{eq:dq_constr})
    must be satisfied. Since the matrix $dh\left(q\right)$ consists of two linearly independent rows, the row vector $Ndh\left(q_{0}\right)\in\mathbb{R}^{1\times3}$
    is nonzero. Furthermore, the orthogonal complement of this row vector is a two-dimensional subspace that coincides with $\mathrm{Im}\left[B\left(q_{0}\right)\right]$.
    Observing that the vector $G$ does not lie in 
    $\mathrm{Im}\left[B\left(q_{0}\right)\right]$,
    we conclude that $Ndh\left(q_{0}\right)G$ is nonzero. Therefore, the equation~(\ref{eq:dq_constr}) can only hold for specific values of
    $\dot{q}$, namely $\dot{q}=\pm\dot{q}_{0}$. However, if we consider $\dot{q}=0\in T_{q_{0}}\mathcal{C}$, the second term of~(\ref{eq:dq_constr}) vanishes, causing the equality to be violated.
    This contradicts the requirement that~\eqref{eq:dq_constr} must hold for all $\dot{q} \in T_{q_{0}} \mathcal{C}$.
    Therefore, we conclude that $\Gamma$ is \textbf{not a controlled invariant manifold}. \qed

    This general result can also be verified by the reader through direct computation using a specific candidate for the VHC, such as
    \begin{equation}
    \label{eq:implicit-vhc-candidate}
      h\left(q\right)=\left(\begin{array}{c}
        z+\frac{1}{2}x^{2}\\
        \psi-\frac{\pi}{2}+\arctan\left(2x\right)
        \end{array}\right)
    \end{equation}
    or any other function $h\left(q\right)$ that satisfies  $h\left(q_{*}\left(t\right)\right)\equiv0$
    and has $\mathrm{rank}\,dh\left(q\right)=2$ on $\mathcal{C}$.
    
  \subsection{Trajectory Stabilizability}
    The arguments above naturally raise the question of the practical realizability of such a trajectory in a physical system, particularly in the presence of model uncertainties. One might question whether the strong requirement of passing the system through the given configuration $q_{0}$ with a precisely specific velocity is feasible, as it may appear unrealistic in practice.
    However, this requirement is a direct consequence of imposing a VHC that constrains the system to move along a prescribed direction. In real-world scenarios, slight deviations from the VHC allow the system to traverse this configuration with a range of possible velocities.

    To demonstrate the existence of an orbitally stabilizing feedback, we adopt the transverse linearization approach~\cite{Banaszuk-1995,Surov-2020}, wherein the dynamics~(\ref{eq:pvtol-dynamics}) are expressed in new local coordinates
    $\tau \in S^1$ and $\rho \in \mathbb{R}^5$, which are diffeomorphic to $(q, \dot q)$ within a tubular neighborhood of the reference orbit.
    Although various coordinate transformations can serve this purpose (see, for example,~\cite{Surov-2020}), we aim to show that the VHC~(\ref{eq:implicit-vhc-candidate}) is also suitable, even though it does not satisfy Definition~\ref{def:VHC}. We define the new coordinates as
    \begin{align*}
      & \tau:=\mathrm{atan2}\left(x,\dot{x}\right),\quad\rho_{1,2}:=h\left(q\right),\quad\rho_{3,4}:=L_{\dot{q}}h\left(q\right),\\
      & \rho_{5}:=\left(x-x_{*}\left(\tau\right)\right)\sin\tau+\left(\dot{x}-\dot{x}_{*}\left(\tau\right)\right)\cos\tau
    \end{align*}
    along with the new control input 
    \[
      w:=u-u_{*}\left(\tau\right).
    \]
    The variables $\rho\equiv\left(\rho_{1},\dots,\rho_{5}\right)^{\top}$
    serve as transverse coordinates, meaning they vanish when the phase space point lies on the reference orbit.
    The coordinate $\tau \in [-\pi, \pi)$ represents
    projection of the phase space point onto the reference trajectory.
    It is straightforward to verify that the transformation $\left(q,\dot{q}\right)\mapsto\left(\tau,\rho\right)$
    is diffeomorphic within a tubular neighborhood of the orbit. This implies the existence of inverse transformation $\left(q,\dot{q}\right)=\mu\left(\rho,\tau\right)$
    allowing the dynamics~(\ref{eq:pvtol-dynamics}) to be expressed as
    \begin{equation}
    \label{eq:transverse-dynamics-1}
      \frac{\partial\mu\left(\rho,\tau\right)}{\partial\left(\rho,\tau\right)}\left(\begin{array}{c}
      \dot{\rho}\\
      \dot{\tau}
      \end{array}\right)=\left(\begin{array}{c}
      \dot{q}\\
      B\left(q\right)u_{*}\left(\tau\right)-G+B\left(q\right)w
      \end{array}\right).
    \end{equation}
    Next, we substitute the transformation $\left(q,\dot{q}\right)=\mu\left(\rho,\tau\right)$
    into the right-hand side of~(\ref{eq:transverse-dynamics-1})
    and solve for $\dot{\rho}$ and $\dot{\tau}$. Since the transformation is diffeomorphic, the Jacobian matrix is invertible, ensuring the validity of this step. Moreover, we observe that $\dot \tau > 0$ within a tubular neighborhood and bounded input $w$, which justifies -- following~\cite{Banaszuk-1995} -- dividing $\dot{\rho}$ by $\dot{\tau}$ to obtain the transverse dynamics in the form of a general nonlinear system:
    \begin{equation}
    \label{eq:transverse-dynamics-2}
      \frac{d\rho}{d\tau}=f\left(\rho,\tau,w\right).
    \end{equation}
    Linearizing~(\ref{eq:transverse-dynamics-2}) with respect to $\rho$ and $w$ yields:
    \[
      \frac{d\rho}{d\tau}=A\left(\tau\right)\rho+B\left(\tau\right)w+o\left(\tau,\rho,w\right),
    \]
    where the function $o\left(\tau,\rho,w\right)$ collects all bilinear, quadratic and higher-order terms in $\rho$ and $w$; the matrix functions
    \begin{align*}
      A\left(\tau\right) &:= \left(\frac{\partial f\left(\rho,\tau,w\right)}{\partial\rho}\right)_{\rho=0,w=0}\quad\text{and}\quad \\
      B\left(\tau\right) &:= \left(\frac{\partial f\left(\rho,\tau,w\right)}{\partial w}\right)_{\rho=0,w=0}
    \end{align*}
    are periodic with period $2\pi$.
    Based on the arguments in~\cite[Theorem~1]{Yakubovich-1986}, the controllability of the LTV system 
    \begin{equation}
      \frac{d\bar{\rho}}{d\tau}=A\left(\tau\right)\bar{\rho}+B\left(\tau\right)w\label{eq:linearized-transverse-dynamics}
    \end{equation}
    over one period implies the existence of an exponentially stabilizing feedback of the form $w\left(\tau,\bar\rho\right)=K\left(\tau\right)\bar\rho$ with a $2\pi$-periodic matrix of feedback gains $K\left(\tau\right):\mathbb{R}\to\mathbb{R}^{2\times5}$.
    Moreover, as shown in~\cite{Leonov-2006}, the exponential stability of the transversally linearized system guarantees the asymptotic orbital stability of the reference trajectory for the original nonlinear system.
    Following these arguments, we propose the feedback control law
    \begin{equation}
    \label{eq:nonlinear-control}
      u(q,\dot q) = u_{*}\left(\tau(q,\dot q)\right) + K\left(\tau(q,\dot q)\right) \rho(q, \dot q)
    \end{equation}
    to achieve orbital stabilization of the reference trajectory~(\ref{eq:ref-traj}).

  \subsection{Numerical Analysis}
    To validate our claims, we conducted a numerical analysis of the linearized transverse dynamics~(\ref{eq:linearized-transverse-dynamics}).
    The controllability Gramian~\cite[Proposition~5.2]{Kalman-1960-optimal-control}
    over one period has the following eigenvalues:
    \[
      \mathrm{eigvals} \, W\left(0,2\pi\right): \quad \left\{ 744.,\,70.7,\,15.3,\,5.16,\,0.0537\right\},
    \]
    confirming the controllability of the LTV system. The feedback gain matrix $K\left(\tau\right)$ was computed using the LQR~\cite{Gusev-2010}, with weighting matrices $R=I_{2\times2}$ and $Q=I_{5\times5}$. 
    The eigenvalues of the monodromy matrix $F_{2\pi}$ for the closed-loop system are:
    \begin{align*}
      \mathrm{eigvals} \, F_{2\pi}: 
       \left\{ 1.38,\,5.27\pm0.758i,\,0.268\pm5.48i\right\} \times 10^{-3},
    \end{align*}
    all of which lie inside the unit circle, ensuring exponential stability.
    To demonstrate the effectiveness of the proposed control law~(\ref{eq:nonlinear-control}),
    we performed computer simulations of the closed-loop nonlinear system with the initial state $q_{1}=\left(0.1,-0.5,0.0\right)^{\top}$, $\dot{q}_{1}=0_{3}$,
    and an integration time step of $10$ ms. The simulation results are presented in Fig.~\ref{fig:pvtol-sim}. As observed, the system state converges to the desired orbit, and all signals remain smooth and bounded. The behavior of transverse coordinates $\rho$ and control input $u$ is presented in Fig.~\ref{fig:pvtol-transverse}.
    \begin{figure}
      \begin{centering}
        \includegraphics[width=7.5cm]{fig/pvtol_phase_coords}
      \par\end{centering}
      \caption{Closed loop system simulation results.}\label{fig:pvtol-sim}
    \end{figure}
    \begin{figure}
      \begin{centering}
        \vspace{4pt}
        \includegraphics[width=7.5cm]{fig/pvtol_transverse}
      \par\end{centering}
      \caption{Evolution of transverse coordinates and control inputs versus trajectory projection $\tau$ in the closed loop system.}\label{fig:pvtol-transverse}
    \end{figure}

  \section{Singular Reduced Dynamics}
  \label{sec:non-regular-vhc}
    We aim to show that the trajectory presented above is not an isolated exception, but rather a representative example within a broad class of admissible solutions. To this end, we briefly recall the method of motion planning based on VHCs~\cite{Shiriaev-2005-constructive-tool,Shiriaev-2006-periodic-planning}.
    As known, according to~\cite[Proposition~2]{Shiriaev-2005-constructive-tool},
    the dynamics of system~(\ref{eq:lag-sys}) under an imposed VHC in parametric form~$q=\phi\left(\theta\right)$
    are governed by the scalar differential equation
    \begin{equation}
    \label{eq:alpha-beta-gamma}
      \alpha\left(\theta\right)\ddot{\theta}+\beta\left(\theta\right)\dot{\theta}^{2}+\gamma\left(\theta\right)=0,
    \end{equation}
    where the scalar coefficients $\alpha\left(\theta\right),\beta\left(\theta\right)$
    and $\gamma\left(\theta\right)$ are defined as
    \begin{align*}
      \alpha\left(\theta\right) & =B_{\perp}\left(\phi\left(\theta\right)\right)M\left(\phi\left(\theta\right)\right)\phi'\left(\theta\right),\\
      \beta\left(\theta\right) & =B_{\perp}\left(\phi\left(\theta\right)\right)M\left(\phi\left(\theta\right)\right)\phi''\left(\theta\right)\\
      & \quad+B_{\perp}\left(\phi\left(\theta\right)\right)C\left(\phi\left(\theta\right),\phi'\left(\theta\right)\right)\phi'\left(\theta\right),\\
      \gamma\left(\theta\right) & =B_{\perp}\left(\phi\left(\theta\right)\right)G\left(\phi\left(\theta\right)\right)
    \end{align*}
    and $B_{\perp}\left(q\right)\in\mathbb{R}^{1\times n}\setminus\left\{ 0\right\} $
    is a left annihilator of $B\left(q\right)$.
    For the PVTOL aircraft considered in the previous section, we had used the VHC
    \begin{equation}
    \label{eq:pvtol-vhc}
      \phi\left(\theta\right)=\left(\theta,-\frac{1}{2}\theta^{2},\frac{\pi}{2}-\arctan2\theta\right)^{\top}
    \end{equation}
    and matrix $B_{\perp}\left(q\right)=\left(\cos\psi,\sin\psi,0\right)$,
    which result in the reduced dynamics:
    \begin{equation}
    \label{eq:pvtol-reduced-dynamics}
      \theta\ddot{\theta}-\dot{\theta}^{2}+1=0.
    \end{equation}
    A reader can verify through direct substitution that the piecewise function
    \begin{equation}
    \label{eq:piecewise-solution-pvtol}
      \theta_{*}\left(t\right)=\begin{cases}
          \theta_{1}\sin\frac{t}{\theta_{1}} & t\in\left[\theta_{1}\pi,0\right)\\
          \theta_{2}\sin\frac{t}{\theta_{2}} & t\in\left[0,\theta_{2}\pi\right]
        \end{cases}
    \end{equation}
    with parameters $\theta_{1}<0<\theta_{2}$, 
    is twice continuously differentiable on the interval $t\in\left[\theta_{1}\pi,\theta_{2}\pi\right]$
    and satisfies equation~(\ref{eq:pvtol-reduced-dynamics}).
    By selecting $\theta_{1}=-1$ and $\theta_{2}=1$,
    we obtain the solution $\theta_{*}\left(t\right)=\sin t$
    and consequently, the trajectory $q_{*}\left(t\right) = \phi(\theta_*(t))$ presented in~(\ref{eq:ref-traj}).
    
    As observed, equation~(\ref{eq:pvtol-reduced-dynamics}) exhibits a singularity
    at $\theta=0$, thereby violating requirements of the Picard-Lindelöf
    existence theorem. However, as demonstrated in~\cite[Theorem~1]{Surov-2018},
    smooth solutions to the reduced dynamics with isolated singularities may still exist, 
    provided that certain additional conditions are satisfied by the coefficients
    $\alpha\left(\theta\right),\beta\left(\theta\right)$
    and $\gamma\left(\theta\right)$. 
    We now recall the formulation of this existence
    \begin{thm}
    \label{thm:singular-solutions-existence}
      Let us consider the differential
      equation~(\ref{eq:alpha-beta-gamma}) with the coefficients 
      $\alpha(\theta), \beta(\theta)$ and $\gamma(\theta)$ defined on an open nontrivial interval $\theta\in I\subset\mathbb{R}$.
      Suppose the following conditions hold:
      \begin{itemize}
        \item
          the coefficients $\alpha\left(\theta\right),\beta\left(\theta\right)$
          and $\gamma\left(\theta\right)$ are smooth functions: $\alpha\left(\theta\right)\in C^{3}\left(I\right)$
          and $\beta\left(\theta\right),\gamma\left(\theta\right)\in C^{2}\left(I\right)$;
        \item
          the function $\alpha\left(\theta\right)$ has a unique zero on the
          interval: $\exists!\,\theta_{s}\in I:\alpha\left(\theta_{s}\right)=0$;
        \item
          the following inequalities hold true:
          \[
            \alpha'\left(\theta_{s}\right)>0,\quad\gamma\left(I\right)>0,\quad\frac{\beta\left(\theta_{s}\right)}{\alpha'\left(\theta_{s}\right)}<-\frac{1}{2}.
          \]
      \end{itemize}
      Then, for any real values $\theta_{1},\theta_{2}\in I$, $\theta_{1}<\theta_{s}<\theta_{2}$,
      and $\dot{\theta}_{1}\geq0$, $\dot{\theta}_{2}\geq0$ there exists
      the smooth function $\theta_{*}\left(t\right)\in C^{2}$ defined on
      the interval $\left[t_{1},t_{2}\right]$ satisfying the equation~(\ref{eq:alpha-beta-gamma})
      and the boundary conditions 
      \begin{align}
        & \theta_{*}\left(t_{1}\right)=\theta_{1},\quad\dot{\theta}_{*}\left(t_{1}\right)=\dot{\theta}_{1},\label{eq:piecewise-solution-general}\\
        \exists t_{s}\in\left(t_{1},\infty\right):\quad & \theta_{*}\left(t_{s}\right)=\theta_{s},\quad\dot{\theta}_{*}\left(t_{s}\right)=\sqrt{-\gamma\left(\theta_{s}\right)/\beta\left(\theta_{s}\right)},\nonumber \\
        \exists t_{2}\in\left(t_{s},\infty\right):\quad & \theta_{*}\left(t_{2}\right)=\theta_{2},\quad\dot{\theta}_{*}\left(t_{2}\right)=\dot{\theta}_{2}.\nonumber 
      \end{align}
    \end{thm}

    \begin{rem}
    \label{rem:periodic-solitions}
      Due to the symmetry of equation~(\ref{eq:alpha-beta-gamma})
      with respect to time inversion $t\mapsto-t$, the function $\theta_{*}\left(-t\right)$
      is also a solution to this equation.
      The reversed in time trajectory
      passes through the phase-space point with $\theta=\theta_{s}$ and $\dot{\theta}=-\sqrt{-\gamma\left(\theta_{s}\right)/\beta\left(\theta_{s}\right)}$.
      If we select $\dot{\theta}_{1}=\dot{\theta}_{2}=0$, then by concatenating the
      two functions $\theta_{*}\left(t\right)$ and $\theta_{*}\left(2t_{2}-t\right)$,
      we obtain a periodic solution with period $2t_{2}-2t_{1}$.
    \end{rem}

    It is easy to see that the reduced dynamics~(\ref{eq:pvtol-reduced-dynamics})
    satisfy conditions of Theorem~\ref{thm:singular-solutions-existence}
    for $I=\mathbb{R}$ and $\theta_{s}=0$. The solution~(\ref{eq:piecewise-solution-pvtol})
    corresponds to (\ref{eq:piecewise-solution-general}) if we take $t_{1}=\frac{\pi\theta_{1}}{2}$,
    $t_{2}=\frac{\pi\theta_{2}}{2}$ and $t_{s}=0$.

    Importantly, the singularity presented in Theorem~\ref{thm:singular-solutions-existence}
    is not removable and cannot be eliminated through a coordinate change or different representations of the VHC. To demonstrate this, consider the following
    \begin{thm}
    \label{thm:singularity-necessity} 
      Let $q_{*}\left(t\right),u_{*}\left(t\right)$
      be a smooth solution to the mechanical system~(\ref{eq:lag-sys}).
      Suppose that, at some time $t_{s}$, the trajectory $q_{*}\left(t\right)$
      satisfies 
      \begin{align}
      \label{eq:singularity-condition}
        \exists \, t_s : & \, \, \dot{q}_{*}\left(t_{s}\right)\ne0\quad\text{and}\nonumber \\
                      & \, \, B_{\perp}\left(q_{*}\left(t_{s}\right)\right)M\left(q_{*}\left(t_{s}\right)\right)\dot{q}_{*}\left(t_{s}\right)=0,
      \end{align}
      where $B_{\perp}\left(q\right)\in\mathbb{R}^{1\times n} \setminus \left\{ 0\right\} $
      is a left annihilator of $B\left(q\right)$. Then, reduced dynamics~(\ref{eq:alpha-beta-gamma})
      corresponding to \textbf{any} smooth parametrization $q_{*}\left(t\right)=\phi\left(\theta_{*}\left(t\right)\right)$
      will necessarily satisfy
      \begin{equation}
      \alpha\left(\theta_{s}\right)=0,\quad\text{where}\quad\theta_{s}\equiv\theta_{*}\left(t_{s}\right).\label{eq:alpha-eq-zero}
      \end{equation}
      If, in addition, $B_{\perp}\left(q_{*}\left(t_{s}\right)\right)G\left(q_{*}\left(t_{s}\right)\right)\ne0$,
      then the reduced dynamics exhibit a non-removable singularity at $\theta_{s}$.
    \end{thm}
    %
    \begin{proof}
      We substitute the expressions 
      \[
        q_{*}\left(t_{s}\right)=\phi\left(\theta_{*}\left(t_{s}\right)\right) \quad\text{and}\quad\dot{q}_{*}\left(t_{s}\right)=\phi'\left(\theta_{*}\left(t_{s}\right)\right)\dot{\theta}_{*}\left(t_{s}\right)
      \]
      into condition~(\ref{eq:singularity-condition}). This yields 
      \[
        \underbrace{B_{\perp}\left(\phi\left(\theta_{*}\left(t_{s}\right)\right)\right)M\left(\phi\left(\theta_{*}\left(t_{s}\right)\right)\right)\phi'\left(\theta_{*}\left(t_{s}\right)\right)}_{=\alpha\left(\theta_{s}\right)}\dot{\theta}_{*}\left(t_{s}\right)=0.
      \]
      Assuming that $\phi\left(\theta\right)$ is smooth and $\phi'\left(\theta_{*}\left(t_{s}\right)\right)\dot{\theta}_{*}\left(t_{s}\right)\ne0$,
      it follows that $\dot{\theta}_{*}\left(t_{s}\right)\ne0$ and therefore
      $\alpha\left(\theta_{s}\right)=0$. Now, since $B_{\perp}\left(q_{*}\left(t_{s}\right)\right)G\left(q_{*}\left(t_{s}\right)\right)\ne0$
      we conclude that $\gamma\left(\theta_{s}\right)\ne0$. This implies
      that the right hand side of the equation 
      \[
        \ddot{\theta}=-\frac{\beta\left(\theta\right)}{\alpha\left(\theta\right)}\dot{\theta}^{2}-\frac{\gamma\left(\theta\right)}{\alpha\left(\theta\right)}
      \]
      is not Lipschitz continuous in a neighborhood of $\theta_{s}$. Hence
      the singularity at $\theta_{s}$ cannot be eliminated.
    \end{proof}

    The singularities in reduced dynamics play a crucial role in the loss of controlled invariance of the set $\Gamma$. 
    For a trajectory $q_{*}\left(t\right)$ of the mechanical system~(\ref{eq:lag-sys}) the vanishing
    of the coefficient $\alpha\left(\theta_{s}\right)$
    in corresponding reduced dynamics under parametrization
    $q_{*}\left(t\right)=\phi\left(\theta_{*}\left(t\right)\right)$
    imposes a specific constraint on the derivative $\dot{\theta}_{s}$.
    In particular, under the assumption that $\theta_{*}\left(t\right)$
    is smooth, the following relation must hold:
    \begin{equation}
    \label{eq:constr-for-dtheta}
      \beta\left(\theta_{s}\right)\dot{\theta}_{s}^{2}+\gamma\left(\theta_{s}\right)=0.
    \end{equation}
    If coefficients $\beta\left(\theta_{s}\right)$
    and $\gamma\left(\theta_{s}\right)$ have opposite signs, then
    the mechanical system at the configuration 
    $q_{s}=\phi\left(\theta_{s}\right)$
    can move along the direction $\phi'\left(\theta_{s}\right)$, but
    only with a specific velocity given by:
    $\dot{q}_{s}=\pm\sqrt{-\gamma\left(\theta_{s}\right)/\beta\left(\theta_{s}\right)}\phi'\left(\theta_{s}\right).$

    A similar result is obtained when considering an implicit VHC $h\left(q_{*}\left(t\right)\right)\equiv0$.
    In this case, the admissible velocity $\dot{q}_{s}$ can be determined by imposing the conditions
    \[
      \frac{d}{dt}h\left(q_{*}\left(t\right)\right)=0\quad\text{and}\quad
      \frac{d^{2}}{dt^{2}}h\left(q_{*}\left(t\right)\right)=0\quad
      \text{at} \quad t=t_{s},
    \]
    which lead to the system of two algebraic equations:
    \begin{align}
    \label{eq:condition-for-dq}
      & dh\dot{q}_{s}=0 \\
      & N L_{\dot{q}_{s}}^{2}h-NdhM^{-1}C\dot{q}_{s}=NdhM^{-1}G\quad\text{at}\quad q=q_{s}  \nonumber
    \end{align}
    where constant $N\in\ker\left[dh\left(q_{s}\right)M^{-1}\left(q_{s}\right)B\left(q_{s}\right)\right]^{\top}\setminus\left\{ 0\right\}$. The existence of $N$ follows from the fact $\ensuremath{B_{\perp}\left(q_{s}\right)M\left(q_{s}\right)\dot{q}_{s}=0}$.
    Substituting $\dot{q}_{s}=\dot{\theta}_{s}V$ with some constant $V\in\ker\left[dh\left(q_{s}\right)\right]\setminus\left\{ 0\right\}$,
    reduces system~(\ref{eq:condition-for-dq}) to a quadratic equation in $\dot{\theta}_{s}$:
    \[
      \left(NL_{V}^{2}h-NdhM^{-1}CV\right)\dot{\theta}_{s}^{2}=NdhM^{-1}G ,
    \]
    which is equivalent to~(\ref{eq:constr-for-dtheta}).
    The condition $B_{\perp}\left(q_{s}\right)G\left(q_{s}\right)\ne0$ in
    Theorem~\ref{thm:singularity-necessity}
    implies that $Ndh\left(q_{s}\right)M^{-1}\left(q_{s}\right)G\left(q_{s}\right) \ne 0$. Consequently,
    the equation either admits two solutions of opposite signs or has no solution at
    all. The former case permits the existence of smooth trajectories
    $q_{*}\left(t\right)$, as demonstrated in section above, but destroys
    the structure of the controlled invariant manifold. Therefore, Definition~\ref{def:VHC}
    excludes from consideration all trajectories described in Theorem~\ref{thm:singularity-necessity}.

  \subsection*{Example: PVTOL Aircraft Trajectories}
    To explore trajectories of the PVTOL aircraft, we propose considering the VHC
    \begin{equation}
    \label{eq:pvtol-vhc-poly}
      \phi\left(\theta\right)=q_{s}+B\left(q_{s}\right)\left(\begin{array}{c}
        k_{1}\\
        k_{2}
      \end{array}\right)\theta+\frac{k_{3}}{2}B_{\perp}^{\top}\left(q_{s}\right)\theta^{2},
    \end{equation}
    where $q_{s} \equiv \left(x_{s}, z_{s}, \psi_{s}\right)^{\top}$; $k_{1}, k_{2}, k_{3} \in \mathbb{R}$ to be defined; and $B_{\perp}\left(q_{s}\right) = \left(\cos\psi_{s}, \sin\psi_{s}, 0\right)$.
    As seen, the reduced dynamics 
    \begin{align}
    \label{eq:alpha-beta-gamma-poly-vhc}
      & \alpha\left(\theta\right)\ddot{\theta}+\beta\left(\theta\right)\dot{\theta}^{2}+\gamma\left(\theta\right)=0,\quad\text{with}\\
      & \alpha\left(\theta\right)=k_{1}\sin\left(k_{2}\theta\right)+k_{3}\theta\cos\left(k_{2}\theta\right), \nonumber  \\
      & \beta\left(\theta\right)=k_{3}\cos\left(k_{2}\theta\right),\quad\gamma\left(\theta\right)=\sin\left(\psi_{s}+k_{2}\theta\right)\nonumber 
    \end{align}
    exhibit a singularity at $\theta=0$, corresponding to the configuration $q_{s}$ of the aircraft.
    Applying Theorem~\ref{thm:singular-solutions-existence}
    to the dynamics~(\ref{eq:alpha-beta-gamma-poly-vhc}), we obtain the following system of inequalities:
    \begin{align*}
      & \alpha'\left(0\right)=k_{1}k_{2}+k_{3}>0,\quad\frac{\beta\left(0\right)}{\alpha'\left(0\right)}=\frac{k_{3}}{k_{1}k_{2}+k_{3}}<-\frac{1}{2},\\
      & \gamma\left(\theta\right)=\sin\left(\psi_{s}+k_{2}\theta\right)>0.
    \end{align*}
    This system is solvable\footnote{Since Theorem~\ref{thm:singular-solutions-existence} offers sufficient but not necessary conditions, one can obtain alternative solutions by multiplying all coefficients in the reduced dynamics by $-1$ and then reapplying the theorem. This yields solutions for $\pi<\psi_{s}<2\pi$. } on the interval $0<\psi_{s}<\pi$ under the conditions $k_{3}<0$ and $k_{1}k_{2}>-k_{3}$.
    Thus, based on Remark~\ref{rem:periodic-solitions}, we conclude the existence of periodic solutions near $q_s$. 
    In particular, the PVTOL aircraft is capable of executing a periodic motion with small amplitude around the configuration $\psi_s = \frac{\pi}{4}$. Simulation results illustrating several such trajectories are available at~\url{https://youtube.com/shorts/EM2qfht1D74}. Notably, none of these trajectories admit a VHC that satisfies Definition~\ref{def:VHC}.

  \section{Concluding Remarks}
  \label{sec:conclusion}
    We have demonstrated, through a specific example, that there exists a class of feasible solutions 
    for mechanical systems that satisfy all conventional conditions except for the controlled invariance 
    of the manifold $\Gamma$ in Definition~\ref{def:VHC}. 
    Moreover, we have shown that a VHC failing to meet this definition can still define 
    transverse coordinates and enables the construction of stabilizing feedback.

    The PVTOL system is not unique in this regard. By applying Theorem~\ref{thm:singular-solutions-existence}, 
    one can find solution for other underactuated systems such as the Pendubot, 
    the Furuta pendulum (as in~\cite{Surov-2018}), the pendulum on a cart, and others, 
    which do not admit the existence of a controlled invariant manifold $\Gamma$.

    We therefore conclude that controlled invariance of $\Gamma$ is not essential.
    Furthermore, restricting the term \textit{feasible} to a narrow class of VHCs can be misleading.
    Under such a definition, a VHC might be labeled ``non-feasible''
    even when it successfully generates fully realizable trajectories -- creating a contradiction in terms.
    To resolve this terminological inconsistency, the requirement of controlled invariance in Definition~\ref{def:VHC}
    should instead be replaced with a condition ensuring the existence of a solution to 
    the mechanical system along which the constraint relation holds.

    We note that other admissible trajectories of mechanical systems also 
    lead to singularities in reduced dynamics,
    yet these fall outside the scope of Theorem~\ref{thm:singular-solutions-existence}. 
    Thus, future studies on VHCs should include a more detailed classification of singularities in reduced dynamics 
    and thereby provide the necessary conditions for feasibility.

  \begin{thebibliography}{10}

    \bibitem{Shiriaev-2005-constructive-tool}
      A. Shiriaev, J. W. Perram and C. Canudas-de-Wit, 
      Constructive tool for orbital stabilization of underactuated nonlinear systems: virtual constraints approach,
      in IEEE Transactions on Automatic Control, vol. 50, no. 8, pp. 1164-1176, Aug. 2005, 
      doi: 10.1109/TAC.2005.852568. 

    \bibitem{Surov-2015}
      M. Surov, A. Shiriaev, L. Freidovich, S. Gusev and L. Paramonov, Case study in non-prehensile manipulation: planning and orbital stabilization of one-directional rollings for the "Butterfly" robot, 
      2015 IEEE International Conference on Robotics and Automation (ICRA), 
      Seattle, WA, USA, 2015, pp. 1484-1489, 
      doi: 10.1109/ICRA.2015.7139385.
  
    \bibitem{Freidovich-2007} 
      L. Freidovich, A. Robertsson, A. Shiriaev, R. Johansson,
      Periodic motions of the Pendubot via virtual holonomic constraints: Theory and experiments,
      Automatica, Volume 44, Issue 3, 2008, 
      Pages 785-791, 
      doi.org: 10.1016/j.automatica.2007.07.011.

    \bibitem{Maggiore-2013}
      M. Maggiore and L. Consolini,
      Virtual Holonomic Constraints for Euler--Lagrange Systems,
      in IEEE Transactions on Automatic Control, vol. 58, no. 4, pp. 1001-1008, April 2013,
      doi: 10.1109/TAC.2012.2215538.

    \bibitem{Consolini-2011}
      L. Consolini and M. Maggiore,
      On the swing-up of the Pendubot using virtual holonomic constrains,
      2011 50th IEEE Conference on Decision and Control and European Control Conference, Orlando, FL, USA, 2011, pp. 4803-4808,
      doi: 10.1109/CDC.2011.6161512.

    \bibitem{Otsason-2019}
      R. Otsason and M. Maggiore,
      On the Generation of Virtual Holonomic Constraints for Mechanical Systems With Underactuation Degree One,
      2019 IEEE 58th Conference on Decision and Control (CDC), Nice, France, 2019, pp. 8054-8060, 
      doi: 10.1109/CDC40024.2019.9029201.

    \bibitem{Buss-2016} 
      B. G. Buss, K. A. Hamed, B. A. Griffin and J. W. Grizzle, Experimental results for 3D bipedal robot walking based on systematic optimization of virtual constraints,
      2016 American Control Conference (ACC), Boston, MA, USA, 2016, pp. 4785-4792, 
      doi: 10.1109/ACC.2016.7526111.
  
    \bibitem{Isidori-1995}
      A. Isidori,
      Nonlinear Control Systems, 3rd Edition, 
      Springer, 1995, Berlin,
      doi: 10.1007/978-1-84628-615-5 

    \bibitem{Consolini-2010}
      L. Consolini, M. Maggiore,
      Virtual Holonomic Constraints for Euler-Lagrange Systems, 
      IFAC Proceedings Volumes, Volume 43, Issue 14, 2010, Pages 1193-1198.
  
    \bibitem{Jankuloski-2012}
      D. Jankuloski, M. Maggiore, L. Consolini,
      Further Results on Virtual Holonomic Constraints,
      IFAC Proceedings Volumes, Volume 45, Issue 19, 2012, Pages 84-89.

    \bibitem{Consolini-2018}
      L. Consolini, A. Costalunga, M. Maggiore,
      A coordinate-free theory of virtual holonomic constraints,
      Journal of Geometric Mechanics, 2018, 10(4): 467-502, 
      doi: 10.3934/jgm.2018018.

    \bibitem{Hauser-1992-pvtol} 
      J. Hauser, S. Sastry, G. Meyer,
      Nonlinear control design for slightly non-minimum phase systems: Applicationto V/STOL aircraft,
      Automatica, Volume 28, Issue 4, 1992, Pages 665-679, ISSN 0005-1098,
      doi: 10.1016/0005-1098(92)90029-F.
  
    \bibitem{Banaszuk-1995}
      A. Banaszuk, J. Hauser,
      Feedback linearization of transverse dynamics for periodic orbits, 
      Systems \& Control Letters, 
      Volume 26, Issue 2, 1995, Pages 95-105, ISSN 0167-6911, 
      doi: 10.1016/0167-6911(94)00110-H.
  
    \bibitem{Surov-2020}
      M. Surov, S. Gusev and L. Freidovich, 
      Constructing Transverse Coordinates for Orbital Stabilization of Periodic Trajectories,
      2020 American Control Conference (ACC), Denver, CO, USA, 2020, pp. 836-841,
      doi: 10.23919/ACC45564.2020.9147533.
  
    \bibitem{Yakubovich-1986}
      V.A. Yakubovich,
      Linear-quadratic optimization problem and frequency theorem for periodic systems,
      Siberian Math. J., 27, 181--200.

    \bibitem{Leonov-2006}
      Leonov G. A.,
      Generalization of the Andronov–Vitt theorem,
      Regular and Chaotic Dynamics,
      2006, Volume 11, Number 2, pp. 281-289.

    \bibitem{Kalman-1960-optimal-control}
      R.E. Kalman,
      Contributions to the Theory of Optimal Control. Matematica Mexicana, 1960, 5, 102-119. 

    \bibitem{Gusev-2010}
       S. Gusev, S. Johansson, B. Kågström, et al,
       A numerical evaluation of solvers for the periodic Riccati differential equation,
       Bit Numer Math 50, 301–329 (2010),
       doi: 10.1007/s10543-010-0257-5.

    \bibitem{Shiriaev-2006-periodic-planning}
      A. Shiriaev, A. Robertsson, J. Perram, A. Sandberg,
      Periodic motion planning for virtually constrained Euler-Lagrange systems,
      Systems \& Control Letters, Volume 55, Issue 11, 2006, Pages 900-907.

    \bibitem{Surov-2018}
      M. Surov, S. V. Gusev and A. S. Shiriaev,
      New Results on Trajectory Planning for Underactuated Mechanical Systems with Singularities in Dynamics of a Motion Generator, 
      2018 IEEE Conference on Decision and Control (CDC), Miami, FL, USA, 2018, pp. 6900-6905,
      doi: 10.1109/CDC.2018.8618669.

  \end{thebibliography}
    
\end{document}
